\documentclass[a4paper,12pt]{article}
\usepackage[top=3cm, bottom=3cm, left=3cm, right=3cm]{geometry}	%dimension de la feuille
\usepackage[utf8]{inputenc}%codage linux
\usepackage[T1]{fontenc}%les fontes
\usepackage[french]{babel}%caractères français
\usepackage{amsmath}
\usepackage{amssymb}
\usepackage{mathrsfs}
\usepackage{hyperref}%liens pdf
\usepackage{listings}%pour mettre du code
\usepackage{color}%pour la couleur dans le code
\usepackage{graphicx}%pour \includegraphics
\usepackage{float}
\usepackage{longtable}
\usepackage[table]{xcolor}
\lstset{breaklines=true}
\usepackage{verbatim}
\usepackage{array}
\usepackage{svg}

%%%%%%%%%%%%%%%%%%%% Presentation %%%%%%%%%%%%%%%%%%%
\title{Informatique Répartie\\Dossier de Conception\\Sujet : Messagerie instantanée 2}
\author{Ingrid FIQUET\\Morgane LEGROS\\Florian MARTIN\\Thibault THÉOLOGIEN\\Youssef ZERHOUNI ABDOU}
\date{\today}


\begin{document}
	\begin{titlepage}
		\vfill
		\begin{figure}
			\includegraphics[scale=0.3]{img/logoINSARouen.png}
		\end{figure}

		\maketitle

		\vfill
		\noindent \hrulefill

	\end{titlepage}


%%%%%%%%%%%%%%%%%%%% Corps %%%%%%%%%%%%%%%%%%%

	\newpage
	\tableofcontents
	\newpage

	\section*{Introduction}

	Ce document concerne la réalisation d'une application de messagerie instantanée.
	Il s’agira d’une messagerie permettant à la fois le dialogue vers un utilisateur unique ou bien vers un ensemble d’utilisateurs.
	La discussion pourra être en audio conférence dans le cas d’un dialogue vers un unique utilisateur.
	Chaque utilisateur pourra dans les paramètres de son compte gérer la configuration du système de filtrage de la messagerie.

	En ce qui concerne les utilisateurs, la messagerie sera accessible par des utilisateurs humains ou bien par des utilisateurs virtuels.
	Les utilisateurs virtuels seront capables d’une discussion simple lors d’un dialogue avec un unique utilisateur.
	Enfin l’administrateur aura lui accès à l’ensemble des comptes utilisateurs et sera le seul autorisé à faire certaines actions telles que bannir un utilisateur par exemple.

	Cette messagerie instantanée sera supportée par tous les OS, accessible via tous les navigateurs et depuis tous types d’appareils (PC, smartphone).

	\newpage

	\section{Conception préliminaire}

	\subsection{Diagramme de modèle du domaine}
	Cette partie présente les diagrammes de modèle du domaine du client et du serveur.
	Le diagramme de Modèle du Domaine serveur montre la structure envisagée pour notre serveur.
	On y trouve donc une entité Serveur chargée de la communication avec les clients.
	Les autres entités représentent la structure des données pour le fonctionnement de notre application.

	\begin{figure}[H]
		\centerline{\includegraphics[width=16.5cm]{../diagrammes/img/modeleDomaineServeur.png}}
		\caption{Diagramme de modèle du domaine côté serveur}
	\end{figure}

	\newpage

	Ce diagramme de Modèle du Domaine présente la structure générale de notre application du côté du client.
	Celui-ci réutilise la plupart des entités présentes dans le modèle du domaine du côté serveur.
	Ainsi, nous avons une cohérence entre les données des deux côtés.
	\begin{figure}[H]
		\centerline{\includegraphics[width=16.5cm]{../diagrammes/img/modeleDomaineClient.png}}
		\caption{Diagramme de modèle du domaine côté client}
	\end{figure}

	\newpage

	\subsection{Diagramme de séquence système}
	Voici un diagramme séquence système correspondant au cas d'utilisation le plus important du projet : envoyer un message
	\begin{figure}[H]
		\centerline{\includegraphics[width=12.5cm]{../diagrammes/img/sequenceSystemeEnvoiMessage.png}}
		\caption{Diagramme de séquence système de l'envoi de messages}
	\end{figure}

	\newpage

	\subsection{Diagramme d'activités de navigation}
	Lorsque l'utilisateur arrive sur l'application, il arrive d'abord sur un "Page de login" où l'utilisateur a le choix entre :
	\begin{itemize}
		\item connexion : rentrer ses identifiants dans "pseudonyme" et "mot de passe" afin de se connecter en appuyant sur le bouton de validation
		\item inscription : créer un compte en cliquant sur "Pas encore inscrit ?".
		Dans ce cas, l'utilisateur arrive sur la "Page d'inscription" où il est possible de rentrer ses identifiants "Pseudonyme", "Adresse mél", "Mot de passe", "Confirmation", "Nom", "Prénom", et "Date de naissance".
		L'utilisateur s'inscrira en appuyant sur le bouton de validation.
		L'utilisateur peut également retourner sur la "Page de login" en appuyant sur le bouton d'annulation.\\
	\end{itemize}

	Lorsque l'utilisateur a passé l'étape de connexion ou d'inscription, il arrive sur la "Page de Messagerie" où l'utilisateur visualise les différentes conversations déjà créées. Sur cette page, différentes possibilités s'offre à lui :
	\begin{itemize}
		\item "Page de conversation" : l'utilisateur peut cliquer sur une conversation déjà existante et dans ce cas il arrive sur une "Page de conversation"
		\item "Page de paramètres" : l'utilisateur peut cliquer sur l'émoticone de paramètres et dans ce cas il arrive sur la "Page de paramètres"
		\item "Page de contacts" : l'utilisateur peut cliquer sur l'émoticone des contacts, à gauche du nom de l'application, et dans ce cas il arrive sur la "Page de contacts"
		\item "Nouvelle conversation" : l'utilisateur peut cliquer sur l'émoticone "+" afin de créer une nouvelle conversation, dans ce cas une pop-up s'ouvrira pour choisir un ou plusieurs utilisateurs, il validera et arrivera sur une "Page de conversation" associée\\
	\end{itemize}

	Lorsque l'utilisateur se situe sur une "Page de conversation", différentes possibilités s'offre à lui :
	\begin{itemize}
		\item envoyer un message en remplissant le champ "Saisie" et cliquer sur l'émoticone envoyer
		\item "Page de paramètres" : de la même manière
		\item "Page de Messagerie" : l'utilisateur peut cliquer sur l'émoticone de retour et dans ce cas il retournera sur la "Page de Messagerie"\\
	\end{itemize}

	\newpage

	Lorsque l'utilisateur se situe sur une "Page de contacts", différentes possibilités s'offre à lui :
	\begin{itemize}
		\item appeler un contact en cliquant sur l'émoticone d'appel.
		Dans ce cas une popup de "Conversation Audio" s'ouvre, l'utilisateur pourra raccrocher en appuyant sur l'émoticone d'appel rouge afin de couper la conversation audio et retourner sur la "Page de contacts"
		\item "Page d'un conversation" : l'utilisateur peut cliquer sur l'émoticone de message pour contacter un contact dans une conversation privée et dans ce cas il arrive sur la "Page de conversation" associée
		\item "Page de paramètres" : de la même manière
		\item "Page de Messagerie" : l'utilisateur peut cliquer sur l'émoticone de Messagerie, à gauche du nom de l'application, et dans ce cas il arrive sur la "Page de Messagerie"\\
	\end{itemize}

	Lorsque l'utilisateur se situe sur la "Page de paramètres", différentes possibilités s'offre à lui :
	\begin{itemize}
		\item modifier ses paramètres en remplissant les champs
		\item retourner sur la page précédente en cliquant sur l'émoticone de retour
		\item se déconnecter en cliquant sur le bouton "Déconnexion"\\
	\end{itemize}

	\begin{figure}[H]
		\centerline{\includegraphics[width=16.5cm]{../diagrammes/img/navigation.png}}
		\caption{Diagramme de navigation}
	\end{figure}


	\subsection{Tableau d’Interaction}
	Dans cette partie nous allons vous présenter toutes les interactions qui interviennent lors de l'envoi d'un message de la part d'un utilisateur à l'aide d'un tableau d’interaction et d'un diagramme d’interaction.

	\textbf{Tableau d’interaction pour envoyer un message :} \\

	Action de début : Vouloir envoyer un message à une personne. \\

	\begin{tabular}{|p{7cm}|p{7cm}|}
		\hline
		Action de l'utilisateur & Action du système\tabularnewline
		\hline

		\hline
		1) Entrer son login et mot de passe (A)  & \tabularnewline

		\hline
		2) Valider  & 3) Vérifier le login et le mot de passe dans la base de données\tabularnewline

		\hline
		4) Sélectionner une discussion commencée ou cliquer sur le "+" pour	démarrer une conversation avec une nouvelle personne ou aller dans les contacts et cliquer sur l’Icône message en face de la personne correspondante & \tabularnewline

		\hline
		5) Envoyer son message  & \tabularnewline
		\hline
	\end{tabular}

~\\

	Action de fin  : Message affiché dans la discussion. \\

	Exception A : Si l'utilisateur n'a pas encore de compte, la première action est de cliquer sur pas encore inscrit, ce qui envoie l'utilisateur sur la page d'inscription où il remplie le formulaire et valide.

	\begin{figure}[H]
		\centerline{\includegraphics[width=12.5cm]{../diagrammes/img/interaction.png}}
		\caption{Diagramme d’interaction}
	\end{figure}

	\newpage

	\subsection{Découpage en packages et les signatures externes de chaque package}
	Les deux diagrammes suivants représentent le découpage en package de notre application.
	Les packages du coté serveur et client sont idépendants même si ils utilisent des classes communes puisqu'il ne sont pas dans le même langage.
	\begin{figure}[H]
		\centerline{\includegraphics[width=16.5cm]{../diagrammes/img/packageServeur.png}}
		\caption{Diagramme de package côté serveur}
	\end{figure}

	\begin{figure}[H]
		\centerline{\includegraphics[width=16.5cm]{../diagrammes/img/packageClient.png}}
		\caption{Diagramme de package côté client}
	\end{figure}

	\newpage

	\section{Conception détaillée}

	\subsection{Diagrammes de classes}
	Les diagrammes de classes présentés dans cette partie se basent sur les diagrammes de modèle du domaine explicités précédemment.
	Ainsi les entités ont été reportées en classes, avec les fonctions qui leur sont associées.
	\begin{figure}[H]
		\centerline{\includegraphics[width=16.5cm]{../diagrammes/img/classesServeur.png}}
		\caption{Diagramme de classes côté serveur}
	\end{figure}

	\newpage

	Ce diagramme de classes représente la partie client de notre application, avec les attributs et méthodes des différentes classes.
	Les getter et setter ne sont pas présents sur celui-ci mais seront présents sur tout les attributs des classes Message, Discussion et Utilisateur.
	\begin{figure}[H]
		\centerline{\includegraphics[width=16.5cm]{../diagrammes/img/classesClient.png}}
		\caption{Diagramme de classes côté client}
	\end{figure}

	%% Détailler par package les élements les constituant.
	%% TODO :
	%% - Préciser les attributs et méthodes de classe de toutes les classes participantes et de les regrouper dans un diagramme de classes.

	%% Les méthodes d’un package qui seront considérée comme non triviales devront être commentées et voir leur fonctionnement détaillé par du pseudo-code.
	%
	% \begin{figure}[H]
	% 	\centerline{\includegraphics[width=16.5cm]{diagClassServeur.png}}
	% 	\caption{Diagramme de classes}
	% \end{figure}

\end{document}
