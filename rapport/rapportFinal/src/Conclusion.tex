%\section{Perspective}

\par Parmi les fonctionnalités principales prévu nous avons réalisé les fonctionnalités suivantes : 

\begin{itemize}
	\item Création de compte, connexion et déconnexion ;
	\item Discussion en ligne ;
	\item Contrôle parentale et filtrage
	\item Création d’une IA en tant qu’utilisateur de la messagerie ;
	\item Image de profil pour chaque utilisateur.
\end{itemize}

Parmi les fonctionnalités que nous n'avons pas réalisé figurent : 

\begin{itemize}
	\item Discussion audio ;
	\item Avatar 3D lors de la conversation audio ;
	\item Affichage d’emoji ;
\end{itemize}

\par Cela s'explique principalement par le fait que nous avons manqué de temps pour la réalisation des discussions audio qui aurait nécessitait beaucoup de temps.

\par De même pour la mise en place d'avatar animé et de l'affichage d'emoji dans les conversation qui n'était pas des fonctionnalité primordiale au projet.  \\



\par En ce qui concerne l’amélioration de notre projet nous pouvons suggéré une amélioration du point de vu de l'IA.

\par En effet, la version actuelle de l'IA est très minimaliste. Elle fonctionne avec une reconnaissance de la question de l'utilisateur au mot près afin d'y associer une réponse. Pour cela le programme de l'IA définit des catégorie de questions de l'utilisateur et de réponses de l'IA par thématique. Ces catégorie sont chacune décris par un fichier xml qui contient tous les réponses de l'IA possible selon la thématique et toutes les questions possibles de l’utilisateur selon la catégorie. Ainsi quand l'utilisateur pose une question le programme reconnaît la catégorie de la conversation et choisi une réponse de façon aléatoire dans le fichier de réponse correspondant à la bonne catégories de l'IA. 

\par Cette solution n'est donc pas efficace pour une vraie conversation car il faut prévoir à chaque fois tous les cas possibles. Actuellement divers solution permettent d'intégrer des chatbots à des sites web en définissant à travers une API, ou un language (par exemple AIML) toutes les régles de fonctionnement de l'IA. Ainsi les plateformes qui proposent ces services mettent à disposition des algorithmes très puissant de réseau de neurone notamment qui permettent de simuler une conversation humaine. Parmi les solution actuelle on peut citer l'api de Google : API.AI qui permet de développer, entrainer et déployer son propre chatbot. 


%\newpage

%\section{Conclusion}